% Start - Hardware/Control Unit/Oscillator Circuit
%write Between the comments


\subsubsection{Oscillator Circuit}
%first create all the reference in bibfile.

	\paragraph{The} oscillator circuit is required for generating clock for \gls{mcu}.Accuracy of timing application is dependent on the accuracy of the oscillator provided in the \gls{mcu}. The oscillators supported by ATmega328P are listed below.
		\begin{itemize}
			\item Low power crystal oscillator
			\item \textbf{Full swing crystal oscillator}
			\item Low frequency crystal oscillator
			\item Internal 128kHz RC oscillator
			\item Calibrated internal RC oscillator
			\item External clock
		\end{itemize}		
		
	\subparagraph{For }
	this application the \gls{mcu} will be in Full swing crystal oscillator mode.
	
	\subparagraph{\textbf{Advantages:}}
		
		\begin{itemize}
			\item 16 MHz.
			\item rail to rail swing.
			\item Drive other clock sources
			\item Less effect of noisy environment 
		\end{itemize}		
	 
	 \subparagraph{Disadvantages}
	 
	 	\begin{itemize}
	 		\item Higher current consumption than Low power crystal oscillator.
	 		\item Operating Voltage of \gls{mcu} becomes 2.7 V to 5.00 V. 
		\end{itemize}	
	
	\subparagraph{Overcoming}
	the disadvantage : Using 230V AC source, so power consumption wont be an issue, \gls{mcu} will be working on 5 V DC which is within the operating voltage defined. 	   	  
		
	\paragraph{Selecting} crystal requires following considerations.
		\begin{enumerate}
				\item \gls{tht} or \gls{smt}.
				\item Load capacitance.
				\item Frequency of operation
				\item Q - factor
				\item \gls{esr}
				\item Frequency Pulling
				\item Drive level
				\item Minimum negative Resistance
				\item Frequency stability
				\item Frequency Tolerance
		\end{enumerate}
			
	\paragraph{\gls{pcb} } design.
	

\cite{AVR042}

% End - Hardware/Control Unit/Oscillator Circuit